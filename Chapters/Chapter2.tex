% Thesis composition

\chapter{Composition of the Thesis}
\label{Chapter2}

This dissertation is divided into chapters as follows.

Chapter~\ref{ch:muscle} describes the McKibben artificial muscle, its general applications
as well as the pros and drawbacks of using this kind of actuator.

Chapter~\ref{ch:model} concerns the process of obtaining a simplified model
from real data through model identification techniques.
After this, a hysteresis component is introduced in the model,
namely the Bouc-Wen model of hysteresis,
to grant a better fit between simulated and real data.
This step is essential so that this model can be
used together with model-based control techniques
so that it is possible to precisely control the muscle's position.
Two different versions of the Bouc-Wen model have been studied and introduced,
and their differences are highlighted in this Chapter.

Chapter~\ref{ch:optimization} introduces the concept of \textit{Evolutionary Algorithm}
(EA) and \textit{Metaheuristic Optimization} (MO).
Three methods belonging to these categories of optimization techniques have been developed
in order to find the best selection of parameters for the hysteresis model, 
and grant the best fit possible.

\begin{itemize}[noitemsep]
	\item \textbf{Evolutionary Algorithms} simulate the process of evolutio
	that characterizes biological systems.
	
	A \textit{Genetic Algorithm}~\cite{fleming2001genetic} (GA) approach
	has been studied and implemented to find a set of parameters
	that could provide a good fit between simulated and experimental values.
	\item \textbf{Metaheuristic Optimization} (MO) algorithm do not guarantee 
	that a globally optimal solution may be found, but are designed to find,
	generate or select a set of parameters that may provide a sufficiently good
	solution to an optimization problem. Metaheuristics sample a set of solutions
	which is too large to be completely sampled. 
	One of the main advantages of this method is that they may be used for a variety
	of problems with high performance.
	
	Two methods belonging to this category have been studied and developed:
	the \textit{Firefly Algorithm}~\cite{yang2010nature}, which mimics the
	behaviour of fireflies, and \textit{Particle Swarm Optimization}~\cite{kennedy2011particle},
	which iteratively tries to improve a candidate solution
	with regard to a given measure of quality.
	
\end{itemize}

The performance of each algorithm, based both on calculation time and fit result,
is compared in Sections~\ref{sec:5.comparison_time} and~\ref{sec:5.comparison_res}.

Chapter~\ref{ch:control} covers the development and application of a PI control,
as well as the tentative development of a model predictive control.

Chapter~\ref{ch:conclusion} lists the conclusions and gives mentions for future works.

A series of appendices follows the chapters.

Appendix~\ref{app:algorithms} contains
all scripts and algorithms code used for this work, all thoroughly described.

Appendix~\ref{app:bouc-wen} covers the Bouc-Wen model of hysteresis.

