\chapter{Conclusions and Future Developments}
\label{ch:conclusion}

McKibben muscles have proven to be of large utility in many fields,
such as rehabilitation, training and actuation of small scale machinery.

The use of tap water as a medium to raise the internal pressure of this kind of muscles
has many advantages: being oil free, they are naturally environment-friendly. Moreover,
the use of water instead of oil makes them safer to use in specific situations.

However, there are drawbacks: water brings problems such as cavitation,
and its lower viscosity makes it harder to control. Furthermore, the McKibben
artificial muscle present a hysteretic behaviour that need to be accounted for
to achieve good control performance.

In this dissertation, the Bouc-Wen model of hysteresis has been proposed
to model the hysteresis of the muscle. This allows to get a very simple,
yet precise model that can be easily controlled using techniques like PID
or MPC control. The Bouc-Wen model has been proposed both in its standard form,
that has been referred as \emph{classic}, and a \emph{generalized} form.
Both versions can express a hysteretic behaviour by use of several
parameters: 4 in case of the classic model, and 8 in case of the generalized one.

Finding these parameters to precisely model a given hysteresis is a hard matter.
While values can be cherry-picked, to achieve highest grades of precision
different methods are needed. In this work, three distinct approaches
have been studied and developed: a Genetic Algorithm, that falls in the category
of Evolutionary Algorithms and that mimics the natural evolution that permeates our world.
Secondly, the Firefly Algorithm that mimics the behaviour of fireflies in the wild,
in its modified version. Lastly, a Particle Swarm Optimization approach where
candidate solutions, called particles, are flown into the solution space and evaluate
it trying to find a global optimum. These last two approaches fall into the category
of Metaheuristic Optimization algorithms.

By using these algorithms it has been showed that it is possible to find parameters
that allow to find a good fit for the hysteresis loop.
Execution times have also been studied to provide a relative rank by timing performance.
This kind of algorithm may be embedded on dedicated hardware
to further speed up the process. This approach may be used
to find parameters for other models and other applications, for example using it
to identify the best gains for a PID controller.

In this work, the control of this type of actuator has been proven to be most effective
with slowly changing reference values. This kind of control may thus be more suitable
for being used in rehabilitation and orthosis, rather than power actuation.

\clearpage

\section{Future Work}

Future work includes:

\begin{itemize}[noitemsep]
	\item Development of Model Predictive Control with the McKibben artificial muscle
	\item Implementation of other Evolutionary Algorithm approaches applied to muscle control
	\item Extension of functionality to the Genetic Algorithm approach to better model the tournament, crossover and mutation steps
	\item Addition of features to the Particle Swarm Optimization approach to attain higher versatility
\end{itemize}