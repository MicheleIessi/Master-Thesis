% Modelling with identification and Hysteresis consideration

\chapter{Modelling of McKibben Artificial Muscles}

In this Chapter, the muscle models used as nominal models for model-based control are derived.

\section{Introduction}

McKibben muscles are actuators used mainly for medical purposes in rehabilitation and welfare. 
The reason is their high flexibility, light weight, low cost, human and environmental friendliness and ease of use.
As mentioned in Chapter~\ref{ch:introduction}, the control of this kind of actuators
is often problematic, due to their inherent high nonlinearity.

Many muscle models have been proposed, both static and dynamic.
One of the most known static models is derived by Chou~\cite{model_chou},
which is based on the equilibrium between the input pressure and the release of energy.
Although the main interest is to obtain the dynamic model to use in control, 
the static one is also reviewed and evaluated because they are used with feed-forward
control applied to rehabilitation devices using the McKibben muscle. 

Dynamic muscle models can be categorized in analysis oriented and control oriented.
Analysis oriented models provide very high accuracy, but they are also very complex.
For this reason they are not much suitable for control purposes.

Control oriented models provide lower accuracy than analysis oriented ones,
but their lower complexity allows them to be used more efficiently for control.

%TODO: add existing research on muscle models

Since tap water driven muscles are simpler than pneumatic ones, the idea is to use
linear system identification and obtain a simple model of the muscle's dynamics.
Being this only a linear model,
it does not take into account the presence of strong nonlinearities,
such as the friction between the braids and the hysteretic behaviour of the muscle.
However, the introduction of a hysteresis model can lead to achieving higher precision
with the model derived from linear system identification.

Through history, several hysteresis models have been developed.
Notable examples are the Maxwell-slip model~\cite{al2005generalized},
the Jiles-Atheron model~\cite{lederer1999parameter}
and the Preisach model~\cite{ge1997generalized}. 
Common interest points of these models are friction of the braided sleeve,
and the hysteresis caused by it, which both add nonlinearities to the model.

These hysteresis models are precise, yet complex in structure.
To overcome this problem, the Bouc-Wen~\cite{bouc}~\cite{bouc_wen} hysteretic model is combined with 
the identified muscle model. Including the hysteretic model raises the number
of parameters that have to be identified for the system, but this is easily
achieved, at first, by trial and error. Later, an adaptive algorithm is developed
to get the best parameters for the muscle, and thus the updated model can be use
as a nominal model for model-based control techniques.

The accuracy of the proposed model is compared to experimental results
by analysing the pressure--displacement characteristics and the hysteresis characteristics.
Moreover, the effects of changing the load of the muscle are studied,
because they may have a great impact upon control performances.

\section{Static Model}


\section{Introduction}
The relationship between the axial contraction force and the pressure difference
amid supply and atmospheric pressure has been reported in different papers~\cite{chou1994static}~\cite{schulte}.

The association is based on the equilibrium between the input work in the McKibben muscle
(\ie~when the fluid is supplied to the inner rubber tube)
and the output work (\ie~when the actuator shortens or elongates
because of the volumetric change associated with the pressure difference).

Figure~\ref{fig:geometric_structure} shows the geometric structure,
which follows the geometric relationships in Equation~\ref{eq:geom_rel}.

\begin{align}
	L = b\cos(\gamma), \quad D = \frac{b \sin(\gamma)}{n \pi}
	\label{fig:geometric_structure}
\end{align}
%\begin{figure}[h]
%	\centering
%	\begin{tikzpicture}
%    \draw (0,0) arc (0:180:1.25 and 0.5);
%    \draw (-2.5,0) -- (-2.5,-3.5);
%	\end{tikzpicture}
%	
%\end{figure}

%TODO: insert muscle figure and braid angle, together with formulas for L and D

\subsection{Static Model of the Muscle}

The input work $W_{in}$ is applied to the muscle when the fluid (liquid or air)
pushes the internal surface of the rubber tube.
This can be expressed as the product of the supply pressure and the change in volume.

\begin{align}
	dW_{in} = \left( P - P_0 \right)dV = P'dV
	\label{eq:input_work}
\end{align}

Where $P$ is the supply pressure, $P_0$ is the atmospheric pressure,
$P'$ is the pressure difference and $dV$ is the volumetric change.

Equation~\ref{eq:muscle_volume} shows the volume of the muscle,
with the assumption that it has cylindrical shape.

\begin{align}
	V = \frac{1}{4}\pi L D^2
	\label{eq:muscle_volume}
\end{align}










