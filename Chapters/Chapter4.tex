% Modelling with identification and Hysteresis consideration

\chapter{Modelling of McKibben Artificial Muscles}

In this Chapter, the muscle models used as nominal models for model-based control are derived.

\section{Introduction}

McKibben muscles are actuators used mainly for medical purposes in rehabilitation and welfare. 
The reason is their high flexibility, light weight, low cost, human and environmental friendliness and ease of use.
As mentioned in Chapter~\ref{ch:introduction}, the control of this kind of actuators
is often problematic, due to their inherent high nonlinearity.

Many muscle models have been proposed, both static and dynamic.
One of the most known static models is derived by Chou~\cite{model_chou},
which is based on the equilibrium between the input pressure and the release of energy.
Although the main interest is to obtain the dynamic model to use in control, 
the static one is also reviewed and evaluated because they are used with feed-forward
control applied to rehabilitation devices using the McKibben muscle. 

Dynamic muscle models can be categorized in analysis oriented and control oriented.
Analysis oriented models provide very high accuracy, but they are also very complex.
For this reason they are not much suitable for control purposes.

Control oriented models provide lower accuracy than analysis oriented ones,
but their lower complexity allows them to be used more efficiently for control.

%TODO: add existing research on muscle models

Since tap water driven muscles are simpler than pneumatic ones, the idea is to use
linear system identification and obtain a simple model of the muscle's dynamics.
Being this only a linear model,
it does not take into account the presence of strong nonlinearities,
such as the friction between the braids and the hysteretic behaviour of the muscle.
However, the introduction of a hysteresis model can lead to achieving higher precision
with with the model derived from linear system identification.

Through history, several hysteresis models have been developed.
Notable examples are the Maxwell-slip model~\cite{al2005generalized},
the Jiles-Atheron model~\cite{lederer1999parameter}
and the Preisach model~\cite{ge1997generalized}. 
Common interest points of these models are friction of the braided sleeve,
and the hysteresis caused by it, which both add nonlinearities to the model.
































