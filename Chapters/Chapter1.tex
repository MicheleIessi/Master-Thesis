% Introduction

\chapter{Introduction}
\label{Chapter1}

% Explain the motivation for the work and the current situation:
% - Background and motivation
% - State of the art + modeling
% - Aim of the study (get a precise and adaptive model of the muscle to use for different loads)

\section{Background and Motivation}

The current state of the art in hydraulic actuators consists almost entirely of
oil driven valves, pistons and motors. However, this kind of actuator cannot be 
used in some particular applications, such as power assist systems and rehabilitation:
they have significant heaviness and rigidity, because of their mechanical structure and
motorization~\cite{mckibben_tondu}. 
In this context, it is problematic to share a robot working space with humans around it, 
and so this kind of actuator cannot be used to actuate orthotics.

McKibben muscles were invented by Joseph L. McKibben to motorize pneumatic art orthotics.
They in general consist of an inner rubber tube enclosed in a braided outer nylon sleeve.
These muscles can be used as actuators of rehabilitation systems due to the following advantages:

\begin{itemize}[noitemsep]
	\item Light weight
	\item High power to weight ratio
	\item High flexibility
	\item Low cost
	\item Low environmental impact
\end{itemize}

However, there are drawbacks: it is well known that the muscle has poor control performance
due to the existence of strong nonlinearities, such as hysteresis and saturation characteristics.
Furthermore, the wear of the materials (nylon sleeve and rubber tube) may cause a shorter lifetime with respect to
other actuators.

%TODO Continue to expand on the background and motivation for the work...

\section{Modelling of McKibben Artificial Muscle}























