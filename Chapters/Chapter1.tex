% Introduction

\chapter{Introduction}
\label{Chapter1}

% Explain the motivation for the work and the current situation:
% - Background and motivation
% - State of the art + modeling
% - Aim of the study (get a precise and adaptive model of the muscle to use for different loads)

\section{Background and Motivation}

The current state of the art in hydraulic actuators consists almost entirely of
oil driven valves, pistons and motors. However, this kind of actuator cannot be 
used in some particular applications, such as power assist systems and rehabilitation:
they have significant heaviness and rigidity, because of their mechanical structure and
motorization~\cite{mckibben_tondu}. 
In this context, it is problematic to share a robot working space with humans around it, 
and so this kind of actuator cannot be used to actuate, for example, orthotics.

McKibben muscles were invented by Joseph L. McKibben to motorize pneumatic art orthotics.
They in general consist of an inner rubber tube enclosed in a braided outer nylon sleeve.
These muscles can be used as actuators of rehabilitation systems due to the following advantages:

\begin{itemize}[noitemsep]
	\item Light weight
	\item High power to weight ratio
	\item High flexibility
	\item Low cost
	\item Low environmental impact
\end{itemize}

However, there are drawbacks: it is well known that the muscle has poor control performance
due to the existence of strong nonlinearities, such as hysteresis and saturation characteristics.
Furthermore, the wear of the materials (nylon sleeve and rubber tube) may cause a shorter lifetime with respect to
other actuators.

%TODO Continue to expand on the background and motivation for the work...

\section{Modelling of McKibben Artificial Muscle}

As already mentioned, the control of McKibben muscles is not easy to achieve.
A PID control solution may be developed, but it has flaws:
the parameters of the controller will have to be tuned different types of muscles
and for various loads. 
This is generally not an acceptable solution for this kind of application.

Thus, model-based control techniques are better suitable for this job. The plant model
needs to be very precise for the control to be effective, and to do so identification techniques are used to get a first linear approximation of the model.
Later, a hysteresis component is added to this linear model, to keep track of the
nonlinearities added by the hysteretic behaviour of the muscle.
Adding a hysteresis component to a linear model makes it more complex, 
so the choice of an appropriate hysteresis modelling technique is crucial to achieve
good control performance.

This allows to get a model having good fit with respect to the real one
and apply a model-based control approach, namely the \textbf{Model Predictive Control} (MPC).
Further details about the developed controller are in Chapter~\ref{Chapter5},
and the theory behind MPC can be found in Appendix~\ref{AppendixA}.

\section{Aim of the Study}

The aim of this dissertation is to get precise control of the displacement for
a tap water-driven McKibben artificial muscle. To do so, a list of steps will be followed:

\begin{enumerate}
	\item \textit{Derivation of a Simple Linear Model} \\ Using linear identification
	techniques, it is possible to map the input pressure to the output displacement,
	and get a precise yet simple linear model of the muscle.
	While in pneumatic artificial muscles it is required to take into account
	the temperature dynamics and the compressibility of air, using water as the muscle
	allows to disregard them. This grants a simpler model of the muscle.
	
	\item \textit{Introduction of a Hysteresis Component} \\ The linear identification
	method grants a very simple yet linear model that does not take nonlinearities into account. 
	The biggest source of nonlinearity for the McKibben muscle is hysteresis.
	A hysteresis component is added to the linear model, 
	so that it will	better follow the actual behaviour of the muscle.
	This step is essential to get good control performance.
	
\end{enumerate}









